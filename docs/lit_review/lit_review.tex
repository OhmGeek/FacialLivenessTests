\documentclass[14pt]{article}
\usepackage[utf8]{inputenc}
\usepackage[english]{babel}
\usepackage{comment}
\usepackage{csquotes}
\title{Literature Review: Facial Authentication System for the Web}
\author{
        Ryan Collins\\
        Supervisor: Andrei Krokhin
}
\date{\today}

\usepackage[
  backend=biber,
  style=alphabetic,
  sorting=ynt
]{biblatex}
\addbibresource{lit.bib}
\begin{document}
\maketitle

\section{Introduction}
\paragraph{Problem Background}
Currently, username and password authentication is commonplace throughout the web. However, username and password
based authentication systems have a number of problems. Some common passwords can be broken using dictionary attacks,
especially if they are a word, or contain a word. Furthermore, the process of shoulder surfing is possible (watching out
for someone's password, and how they type it).

An easy to use system is necessary to remove the choice from the user (in terms of password), relying on the user being automatically
detected, and several confirmation methods to ensure the user is indeed who they say they are (and not just someone spoofing the system).

\paragraph{Areas of Research}


\section{Definitions}


\section{Important Issues of Identified Themes}
% This is the meat and potatoes. We need to answer each question proposed under the 'areas of research' section



\section{Proposed Direction of Project}
% Here is where we specify how we proceed with the project
Therefore, there is a need for a system that encorporates facial liveness, facial recognition and various other
extra security measures together, in a system that is secure for web-based authentication. By creating a service
accessible via an API, these system can be used both for web, as well as for IoT devices, which don't necessarily require
all the security measures of our system, but they're certainly welcome.
\section{Conclusion}
% The conclusion.
\section{References}\label{references}
\printbibliography
\end{document}