\documentclass{article}

\begin{document}
% Research into:
% - Eigenfaces
% - Sparse Low Rank Bilinear Discriminative Model
% - What are liveness tests
% - How can we test a system that we create?
% - What test data is available? ROSE-Youtu Face liveness detection dataset
% - Deep Learning based face liveness detection in videos
% - Cameras used: mobile cameras (most methods use average resolution images from webcams or camera phone)
% - "Deep learning based face liveness detection in videos" https://ieeexplore.ieee.org/document/8090202/
% - Ideas: texture based - an image of a face will have the facial textures, while the texture of a piece of paper is very different.
% - Video vs Photo. Wtih a video, you don't just use the subject image, but also factor in movement and changes.
    \paragraph{Title}
        Facial Authentication System for the Web
    \paragraph{Project Type}
        Computer Vision, Image Processing and Security
    \paragraph{Description}
        - Traditional username and password focused approaches to authentication have drawbacks (such as password leaks)
        - Applying a biometric based approach to web security could improve account security if done correctly.
    \paragraph{Preliminary Preparation}
        \begin{itemize}
            \item Existing biometric web-based authentication methods, how do they work, what are their benefits/drawbacks?
            \item What spoofing methods could be undertaken, and how can we prevent these?
            \item What are the privacy concerns regarding a facial recognition approach, and how can these be mitigated?
            \item How can this be integrated into a web service?
            \item OAuth Authentication - how could facial recognition play into an existing OAuth authentication method?
        \end{itemize}
    \paragraph{Minimum Objectives}
        \begin{itemize}
            \item Server that accepts an image as input via a Web Request, and returns a unique token based on that image.
            \item Generating a token based on facial structure, resilient to background and lightness changes.
        \end{itemize}
    \paragraph{Intermediate Objectives}
        \begin{itemize}
            \item Liveness tests - how can we ensure the user input image is of a person, and not a printout of someone's face?
            \item Scalable system - providing a service layer which is usable by many other users for a variety of uses, and that can scale up if required.
        \end{itemize}
    \paragraph{Advanced Objectives}
        \begin{itemize}
            \item Preventing replay attacks - preventing someone from intercepting someone's facial image, and using it to gain access
        \end{itemize}
    \paragraph{References}
        \begin{itemize}
            \item JSON Web Tokens, and their application
            \item Keras (https://keras.io)
        \end{itemize}
\end{document}